\chapter{Introduction}

As CPU performance and memory capacity grow, it becomes more easy for developers
to focus on the efficiency of the development process instead of the efficiency of
the actual software. As a result, applications get more bloated with features that
are implemented in a needlessly complicated manner, that the end-user may not even utilize.
This is not only a performance loss, but also a potential security hole due to the
increased attack surface of the generated binary file(s).

\section{Software bloating}
Software bloating essentially represents the addition of unnecessary code in an application \cite{explor_bloat}.
Some examples of practices through which an application can get bloated are:
\begin{compactitem}
	\item[$\bullet$] using containers in an improper way
	\item[$\bullet$] bad specification of the product
	\item[$\bullet$] poor deisgn choices
\end{compactitem}
Software bloat can be classified into two major types \cite{bloating_article}:
\begin{compactitem}
	\item[$\bullet$] memory bloat is generally caused by inefficient usage of containers or data structures
	\item[$\bullet$] execution bloat is caused by unnecessary operations
\end{compactitem}
Both of these types of bloating are undesirable and can have a serious impact on the
performance of a program, although they can affect different systems in different ways.
For example, memory bloating can be worse on an embedded system that only does simple
data processing but has a very limited amount of memory.

\section{Purpose}
The purpose of this project is to create a tool that can be used to
analyse the level of bloating for iOS apps and determine which code can be removed in
certain use cases. Furthermore, the user will be able to use the tool to rewrite
the binary to turn off those unwanted or useless features in those apps.

As presented in the introduction of this chapter, reducing the level of bloating
is important for improving the performance of the application and reducing the
risk of it becoming compromised via a security hole. In particular for iOS apps,
debloating them can lead to extended battery life, due to reduced resource usage 
and extra storage space.

There are certain debloating solutions proposed currently, however they are insufficient
for the purposes of this project, as is explained in the section below.

\section{State of the Art}
The debloating solutions currently available are not enough to accomplish the
purposes of this project. Tools like Resurrector \cite{resurrector} (profiles
object lifetime information) and LeakChecker \cite{leak_checker} (tool for checking
memory leaks in managed languages) have too specific targets (Java) in order
to be used for the purposes of this project.

There are two approaches when analysing code for debloating:
\begin{compactitem}
	\item[$\bullet$] static analysis
	\item[$\bullet$] dynamic analysis
\end{compactitem}

Dynamic analysis is done at runtime and is generally used for profiling the application,
in order to learn bloat patterns or determine what portion(s) of code are executed
by a certain feature \cite{bloating_article}. An application of dynamic analysis
is called dynamic slicing. This technique is used in order to determine what are the exact
instructions that produce a certain output given an execution of the program.
A state of the art variant of dynamic slicing is "Abstract Dynamic Slicing" \cite{dynamic_slicing},
which improves regular dynamic slicing by severely reducing the amount of data analysed
(the slicing is done over bounded abstract domains).

Static analysis is done offline and can be used for identifying the semantics of certain
operations. For example, static analysis has been used in order to identify container semantics
 \cite{container}. This has been used in order to detect inefficiencies when using container
operations.

In general, static and dynamic analyses are used in conjunction with one another in order
to detect and eliminate the bloated areas of a program.

\section{Motivation}
Bloating is becoming increasingly prevalent nowadays. A look at the size of the UNIX true shell command
reveals that it has increased by over 300\% in only 4 years (from 8377 bytes in 2010 to 27168 bytes in 2014) \cite{feast_bloat_mitigation}.
That is a big increase, especially considering that the only feature this shell command is actually used for is to return a true value.

Moreover, bloating effectively prevents an application from running at maximum efficiency. Scalability
and efficiency are two very important properties for a program and by debloating an application
those two properties can be maximized, therefore research in this area should be a priority.

Finally, the current state of the art does not offer a good solution for automatically debloating
iOS apps. The only way to debloat an iOS app at the moment is to manually
check its source code for bloating causes, as the existing automatic solutions are too specific
to be used for iOS apps.

\section{Objectives}
The main objective of this project is to create a tool that can be used to tailor the functionality
of an iOS app according to what its user wants. This is achieved by debloating the app of the unused
features or the features that are irrelevant to what the user wants.

This objective can be broken down into smaller objectives:
\begin{compactitem}
	\item[$\bullet$] find out what are the particularities of an iOS app binary by analysing it
	\item[$\bullet$] determine what code can be removed from that iOS app given the information found at the previous step
	\item[$\bullet$] rewrite the binary iOS file with the bloated code removed
\end{compactitem}

\section{Use Cases}
This tool will be used to removed unwanted features form an iOS app binary file and thus
reduce the level of bloating. This is useful to any user who wants to:
\begin{compactitem}
	\item[$\bullet$] minimize an application's attack surface by reducing its size
	\item[$\bullet$] increase an application's performance
	\item[$\bullet$] reduce the resource usage on the phone
\end{compactitem}

As seen from the points above, this tool benefits users who want to optimize and secure
their apps.

\section{Building Blocks}
The basic building blocks for this project are:
\begin{enumerate}
	\item Binary Analysis Platform (bap) - used for analysing a binary file's properties
	\item radare2 - reversing framework, used for determining a function's offset in a file
	\item Keystone Engine - assembler, used to reconstruct the binary file
\end{enumerate}
In addition to these tools, iNalyzer was also used to inspect the iOS binaries beforehand,
in order to find particularities.
This tool is written in Python, since it makes the process of writing code and testing it very
easy and all of the tools cited above (except iNalyzer) have Python bindings.

This tool has been tested on Mac OS X version 10.12.4.

\section{Architectural Overview}
The architecture of the project is not complicated, as seen in the figure below.

\insfigshw{arch.pdf}%
	{Software Architecture Overview}%
	{Software Architecture Overview}%
	{fig.arch_overview}{1}

The two main components of the project are described below.
\subsection{iOS Binary Analyser}
The purpose of this component is to scan the binary file for particularities in order
to find potential features that bloat the application. This is accomplished using the
Binary Analysis Platform. The component looks through a database of patterns built beforehand
using iNalyzer and checks whether the current binary file matches any of them. The matched
patterns are sent to the Debloater.

\subsection{Debloater}
This component receives the patterns and the original file from the binary analyser as well as
the user command. The user is shown a list of all the patterns found along with the features that
they enable and chooses the ones he wants removed from the binary. After that, the debloater
removes the code associated with the patterns, recalulates the address offsets in the file and
writes the new, debloated binary.

\section{Report 1 Conclusions}
Currently, the debloater of the project has the capability of replacing a function's body with nop's, given its name.
Going forward, the project will advance in 3 directions:
\begin{enumerate}
	\item Analyze iOS binaries with iNalyzer in order to find particularities about their feature creep.
	\item Improve the debloater so that it completely removes the given function body. This involves
	recalculating the offsets in the binaries and will be particularly tricky for indirect function calls or jumps.
	\item Research for a way to automatically map code to a feature, instead of relying on a database.
\end{enumerate}

The short-term plan is for the tool to have an internal databse of patterns for certain features through
which it will search when analysing a binary file.
The long-term goal is for the iOS Binary Analyser in the project to be completely autonomous of said database,
and being able to recognize automatically which code determines a certain feature.